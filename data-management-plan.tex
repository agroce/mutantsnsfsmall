%%%%%%%%%%%%%%%%%%%%%%%%%%%%%%%%%%%%%%%%%%%%%%%%%%%%%%%%%%%%%
\documentclass{article}

\begin{document}

\begin{center}
{\Large\sf\textbf{SHF: Medium: A Pathway for Combining Formal, Static,
    and Dynamic Analysis of Real-World Embedded Systems: Data Management Plan}}
\end{center}

The data in the proposed project is primarily either source code for
the tools developed to implement the algorithms and ideas involved or
actual models and annotated code associated with the case study.  The source code will be released either in a project GitHub repository to be created, or incorporated into existing open source projects (e.g., DeepState and Frama-C).  Models and annotations will be released to the SEGA GitHub repository.  In general, this project relies on already open source systems, and exists primarily to extend those systems.  Any experimental data should be reproducible using the code artifacts, but will also be published in a condensed form in the repositories.

Curricular
materials and documentation associated with any tools will also be stored in
an open source repository, since in this project the primary
educational benefits are linked to the use these tools.  Using GitHub automatically provides us with excellent backup
and archiving for the code and curricular material products of the
project.

We have no unusual format or metadata requirements; UPPAAL and other
tools involved use textual format that are easy to store and read;
primarily, in fact, they use textual source code.
We do not anticipate the need to work with sensitive or confidential
information; no extraordinary practices are required in order to
conduct this research.

\end{document}
