%%%%%%%%%%%%%%%%%%%%%%%%%%%%%%%%%%%%%%%%%%%%%%%%%%%%%%%%%%%%%
\documentclass[11pt]{article}
\usepackage[top=1in,bottom=1in,left=1in,right=1in]{geometry}

\begin{document}

\begin{center}
{\Large\sf\textbf{Data Management Plan }} 
\end{center}

% Plans for data management and sharing of the products of research. Proposals
% must include a document of no more than two pages uploaded under "Data
% Management Plan" in the supplementary documentation section of FastLane. This
% supplementary document should describe how the proposal will conform to NSF
% policy on the dissemination and sharing of research results (see Chapter
% XI.D.4), and may include:

As illustrated in the Results of Prior support and biographical sketches, the
PIs have sustained track records of following NSF's policy of prompt publication
of sponsored research data.  In addition, they actively share and communicate
results with the scientific community in conferences and via various synergistic
activities.

All data and tools resulting from this research will be made readily available
through the PIs' websites, related project websites, publicly available GitHub
repositories, technical reports, and peer-review publications.  The produced
implementation code will be open-source, and released under open source
licenses.  Human studies will be designed, in cooperation with institutional
IRBs, to protect the confidentiality and privacy of subjects, and resulting data
will be suitably salted and anonymized prior to analysis or release.

% * the types of data, samples, physical collections, software, curriculum
% materials, and other materials to be produced in the course of the project;

\paragraph{Types of Data.} Data that will be collected as part of this research
include: (1) new algorithms for mutation testing and its constituent components,
e.g., predictors of mutant utility, universal mutant generators, techniques for
feedback elicitation and analysis, and novel diversity and ranking measures, (2)
code that implements the new algorithms in the form of tooling and prototypes,
(3) experimental results from running code or prototypes over open source code
repositories to evaluate efficacy, and (4) designs, surveys, lab study
protocols, and anonymized responses corresponding to human studies assessing
human factors and the utility and usability of mutation-driven development.  The
proposed research will also produce publications for peer review and
dissemination.

% * the standards to be used for data and metadata format and content (where
% existing standards are absent or deemed inadequate, this should be documented
% along with any proposed solutions or remedies);

\paragraph{Data and Metadata standards}
The algorithms, code, and analysis data generated by this project will comprise
only publicly available, non-confidential information. We will implement
algorithms, execution platforms, and support scripts in mainstream programming
languages.  All source code, data resulting from analysis of source code, and
other experimental data will be released under an open source license, such as
the MIT or BSD Licenses.  We have no unusual format or metadata requirements; in
fact, we will mostly rely on textual source code and automatically extracted
vector representations of mutants of code (and mutant results).

Curricular materials and documentation associated with any tools will also be
stored in an open source repository, since in this project the primary
educational benefits are linked to the use of these tools.  Using GitHub
automatically provides us with excellent backup and archiving for the code and
curricular material products of the project.

Our publications generally take the form of refereed conference or journal
publications which are archived in digital and print libraries in, e.g., PDF
format. Any technical reports will be made available on archival NAU or CMU web
servers.

% * policies for access and sharing including provisions for appropriate
% protection of privacy, confidentiality, security, intellectual property, or
% other rights or requirements;

\paragraph{Practices and Policies for Data Access and Sharing}
Source code developed for this project will be released under a suitable open
source license, either in a project GitHub repository to be created, or
incorporated into existing open source projects (the {\bf universalmutator},
{\bf Comby}). We will suitably document the open source code using READMEs and
provide build and supportive reproduction and build scripts as necessary to
support dissemination.  Experimental data should be reproducible using the code
artifacts, along with its associated documentation, but will also be published
in a condensed form in the repositories.

All participants in this proposal will conduct research and publish the results
of their work. Papers will be published in peer-reviewed scientific conferences
and journals (in English); preliminary results may be published in technical
reports.  All information posted on the project website will be maintained, and
updated as appropriate, for a minimum of five years after the project's end
date. Beyond the data posted on the website, primary data, samples, physical
collections, and other supporting materials created or gathered in the course of
work will be shared with other researchers, at no more than incremental cost and
within a reasonable time.  

Software developed during the project will be accessible to the community
through the project websites and web-based source control repositories. Software
and test cases will be developed and stored in a publicly accessible revision
control system, such is commonly used in open-source development projects.

In summary, we will use best practice, open and transparent software development
processes, to promote software sharing and reuse.

\paragraph{Protection of potentially sensitive data}
Human survey and lab study data will be salted and anonymized immediately for
the protection of subjects.  Audio recordings associated with any lab studies
conducted will be transcribed and anonymized for coding, qualitative analysis,
and release, with subject permission; audio or video files will be stored on
secure institutional servers.  PI Le Goues has experience conducting human lab
studies and surveys, and will continue to follow all applicable regulations
while working with the CMU IRB to protect the rights and interests of human
subjects.

The project does not currently target proprietary or confidential data, and
instead targets source code, tests, and defects in publicly-available only
repositories.  We do not anticipate the need to work with proprietary or
confidential information; no extraordinary practices are required in order to
conduct this research.  As such, no further plans for protecting such sensitive
data are necessary.

% * policies and provisions for re-use, re-distribution, and the production of
% derivatives; and
% * plans for archiving data, samples, and other research products, and for
% preservation of access to them.  Data management requirements and plans specific
% to the Directorate, Office, Division, Program, or other NSF unit, relevant to a
% proposal are available at: http://www.nsf.gov/bfa/dias/policy/dmp.jsp. If
% guidance specific to the program is not available, then the requirements
% established in this section apply.


\paragraph{Provisions for Reuse, Redistribution, and Preservation of Access}
All artifacts will be suitably documented. Beyond their open-source release,
additional copies of all data and code will be archived on departmental backup
systems to further ensure their preservation. In addition, papers that present
results based on our research will be made available via the PIs' individual web
pages.

Beyond transcription of audio data from any lab studies, no further
transformations will be necessary to prepare data for preservation/data
sharing. All written reports will be stored and all data will be kept for seven
years beyond the project's end date.  Using GitHub automatically provides us
with excellent backup and archiving for the code and curricular material
products of the project.

\end{document}
