%%%%%%%%%%%%%%%%%%%%%%%%%%%%%%%%%%%%%%%%%%%%%%%%%%%%%%%%%%%%%
\documentclass{article}

\begin{document}

\begin{center}
{\Large\sf\textbf{Collaborative Research: SHF: Small: Feedback-Driven Mutation Testing for Any Language}}
\end{center}

The data in the proposed project is primarily either source code for
the tools developed to implement the algorithms and ideas involved or
results of experiments.  The source code will be released either in a
project GitHub repository to be created, or incorporated into existing
open source projects (the {\bf universalmutator}).  Any experimental data should be reproducible using the code artifacts, but will also be published in a condensed form in the repositories.

Curricular
materials and documentation associated with any tools will also be stored in
an open source repository, since in this project the primary
educational benefits are linked to the use of these tools.  Using GitHub automatically provides us with excellent backup
and archiving for the code and curricular material products of the
project.

We have no unusual format or metadata requirements; in fact, we will
mostly rely on textual source code and automatically extracted vector
representations of mutants of code (and mutant results).
We do not anticipate the need to work with sensitive or confidential
information; no extraordinary practices are required in order to
conduct this research.

\end{document}
