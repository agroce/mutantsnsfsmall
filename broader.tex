\section{Broader Impacts}

\paragraph{Improving Software System Reliability:} A key element of
broader outreach will be to report bugs discovered during testing
experiments, and contribute improved test suites to critical open
source projects.  To that end, this proposal will primarily target real world
systems in experiments, in hopes of improving their quality, and
the quality of their testing.  Infrastructure developed in preliminary
workincludes automated testing for
the Linux kernel RCU module, Google and Mozilla JavaScript engines, a
variety of C compilers (including GCC and LLVM), YAFFS2 \cite{yaffs2}
and other file systems, Google's Go compiler, a large set of Unix
utilities, and a large number of Python libraries (including some of
the most widely used libraries, and key scientific and numeric
analysis packages).  In previous work, discussions with working test
engineers at Mozilla, Google, and NASA have significantly informed the PI's
research efforts, and this is likely to continue.  In the long term,
a mutation-driven development paradigm might result in
easier development of critical software components in tandem with an
extremely high-quality, specification-defining automated test suite.

\paragraph{Education and Outreach:}
The proposed research yields several opportunities for enhancing CS
education, recruiting new CS majors, and retaining CS students,
particularly members of underrepresented groups.  
PI Groce will work with the NAU Student ACM Chapter to present a
series of ``excursions in testing'' that introduce automated testing
to students, using feedback-driven mutation testing  and
mutation-driven-testing on real code, including code from
media player libraries.  The work of Guzdial
\cite{Guzdial} has shown that media computation is a
potentially effective way to both recruit and retain female and
under-represented minority students in computer science.
