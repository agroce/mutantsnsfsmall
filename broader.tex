\section{Broader Impacts}

\paragraph{Improving Software System Reliability:} A key element of
broader outreach will be to report bugs discovered during testing
experiments, and contribute improved test suites to critical open
source projects.  To that end, this proposal will primarily target real world
systems in experiments, in hopes of improving their quality, and
the quality of their testing.  Infrastructure developed in preliminary
work includes automated testing for
the Linux kernel RCU module, Google and Mozilla JavaScript engines, a
variety of C, smart contract, and Go compilers (including GCC and LLVM), YAFFS2~\cite{yaffs2}
and other file systems, a large set of Unix
utilities, and critical Python libraries (including key scientific, ML, and numeric
analysis tools).  Discussions with working test
engineers at Mozilla, Google, Trail of Bits, and NASA have significantly informed the PIs'
research efforts, and this is likely to continue.  PI Groce is
discussing plans for incorporating advanced automated
test generation into NASA's open source F Prime flight
architecture~\cite{fprime,fprimerepo}, 
and F Prime components are a likely target
for evaluation efforts in this project.  Such efforts will
result in a documented process for incorporating feedback-driven
mutation testing and MDD into flight software development.
Better, cheaper, high-quality testing for small budget CubeSat~\cite{CubeSat}
missions could lead to advances in various fields, especially Earth
observation and space-based physics.  CubeSat focuses on providing a
low-cost way for educational institutions and non-profits to conduct space-based
research, and thus is related to education and outreach.
In the long term,
an MDD paradigm might result in
easier development of critical software in tandem with an
extremely high-quality, specification-defining automated test suite.
The existence of such suites might make modifying critical systems
easier, since the in-place test suite would be likely to identify even
subtle problems introduced during changes.  With the growing
impact of embedded, cyberphysical, and Internet-of-Thing systems on
the physical world, this has potential safety benefits for the general public.

\paragraph{Education and Outreach:}
The proposed research yields several opportunities for enhancing CS
education, recruiting new CS majors, and retaining CS students,
particularly members of underrepresented groups.  
PI Groce will work with the NAU Student ACM Chapter to present a
series of ``excursions in testing'' that introduce automated testing
to students, using feedback-driven mutation testing and Mutation-Driven-Development (MDD) on real code, including code from
media player libraries.  The work of Guzdial
\cite{Guzdial} has shown that media computation is a
potentially effective way to both recruit and retain female and
under-represented minority students in computer science.

\paragraph{Undergraduate research mentorship:} 
The PIs are fundamentally committed to widening the pipeline of students
interested in and equipped to pursue a research career. PI Le Goues is
co-director of the REUSE@CMU summer program (funded in part by NSF CNS-1852260,
on which she is a Co-PI). The site trains students in
all elements of research, and specifically seeks students representing
underserved demographic groups, early in their undergraduate careers, and at
schools without traditional research opportunities.  So far, the REUSE program
has resulted in undergraduate research by 118 total students, of whom 68 (57\%)
identified as women; 31 (26\%) were drawn from racial and ethnic groups that are
under-represented in computing (some are both).  The students have led or deeply contributed to more than 75 conference publications; our program has produced 11 winners of ACM student research competitions. More than 80\% of the graduated
students are now doing research full-time, either in graduate school or at
national or corporate research labs.

Undergraduate researchers could contribute to multiple areas of the proposed
work, from design and implementation to evaluation. For example, undergraduates
Zhen Yu Ding (University of Pittsburgh '21) and Yiwei Lyu (Carnegie Mellon
University '21) designed and evaluated the study of diversity metrics for
novelty-promoting search-based software engineering that informs some of the
proposed work in novelty measurement; as an REU summer student, David Widder
(University of Oregon '17, now a CMU PhD student) designed and conducted half of
the talk aloud studies in the qualitative work PI Le Goues conducted with
collaborators on the challenges of framework
debugging~\cite{frameworkDebugging}.  We will continue to integrate
undergraduates both in the summer program and throughout the school year in the
proposed project.
