\section{Broader Impacts}

\paragraph{Improving Software System Reliability:} A key element of
our approach is to focus on realistically deployable techniques.  Part
of this effort is concentrated in our internal effort to apply our
methods to the SEGA project.  However, we also plan
to aim for early integration with NASA's FPrime~\cite{fprime,fprimerepo}
open source
flight software architecture and platform; PI Groce is already in
discussion with engineers at NASA's Jet Propulsion Laboratory, and
engaged in producing tests for the FPrime autocoder using DeepState.
This integration will allow our
methods to be applied to CubeSat missions (and other flight software
systems), leading to improved reliabliity for low-budget space-based
scientific efforts.  We expect, in the long run, that our approaches
will lead to more reliable and robust development in many embedded and
cyberphysical systems domains, and contribute to a more secure and
reliable Internet of Things.  One key goal of this project is to
increase the synergy between formal modeling, heavyweight static
analysis, and advanced dynamic analysis using automated test
generation tools; currently, engineers in real-world projects seldom
use \emph{any} of these approaches.  One barrier to entry is that it
is seldom clear, in a particular project, which of these approaches
will provide the most benefit; formal modeling can detect design
flaws, but seldom provides any help with the most frequent source of
failure, implementation-level flaws; static analysis either tends to
miss bugs, overwhelm users with false positives, or, when extended to
proving correctness, simply fail to prove many components; and,
finally, dynamic analysis is ad hoc and misses subtle flaws.  By
increasing the synergy of these methods, we hope to make use of
\emph{all three methods} in conjunction much more economical (and less
error prone), greatly enhancing the likelyhood of a
payoff in terms of fault detection and prevention.

\paragraph{Education and Outreach:}
The proposed research yields several opportunities for enhancing CS
education, recruiting new CS majors, and retaining CS students,
particularly members of underrepresented groups.  In addition to the
activities discussed at length in the Broadening Participation in Computing plan,
PI Groce will work with the NAU Student ACM Chapter to present a
series of ``excursions in testing'' that introduce automated testing
to students, using DeepState to find bugs in real world code, including code from
media player libraries; in advanced meetings, integrating DeepState
with \framac will be demonstrated as well.  The work of Guzdial
\cite{Guzdial} has shown that media computation is a
potentially effective way to both recruit and retain female and
under-represented minority students in computer science.
