\section{Broader Impacts}

\paragraph{Improving Software System Reliability:} A key element of
broader outreach will be to report bugs discovered during testing
experiments, and contribute improved test suites to critical open
source projects.  To that end, this proposal will primarily target real world
systems in experiments, in hopes of improving their quality, and
the quality of their testing.  Infrastructure developed in preliminary
work includes automated testing for
the Linux kernel RCU module, Google and Mozilla JavaScript engines, a
variety of C compilers (including GCC and LLVM), YAFFS2 \cite{yaffs2}
and other file systems, Google's Go compiler, a large set of Unix
utilities, and a large number of Python libraries (including some of
the most widely used libraries, and key scientific and numeric
analysis packages).  In previous work, discussions with working test
engineers at Mozilla, Google, and NASA have significantly informed the PI's
research efforts, and this is likely to continue.  PI Groce is
currently discussing plans for incorporating more advanced automated
test generation into NASA's open source F Prime flight
architecture~\cite{fprime,fprimerepo}, 
and F Prime components, such as the auto-coder, are a likely target
for evaluation efforts in this project.  Such efforts are planned to
result in a documented process for incorporating feedback-driven
mutation testing and MDD into flight software component development.
Better, cheaper, high-quality testing for small budget CubeSat~\cite{CubeSat}
missions could lead to advances in various fields, especially Earth
observation and space-based physics.  The CubeSat initiative focuses on providing a
low-cost way for educational institutions and non-profits to conduct space-based
research, and thus is also related to education and outreach
activities.
In the long term,
a mutation-driven development paradigm might result in
easier development of critical software components in tandem with an
extremely high-quality, specification-defining automated test suite.
We hypothesize that such suites might make modifying critical systems
easier, since the in-place test suite would be likely to identify even
very subtle problems introduced during changes.  With the growing
impact of embedded, cyberphysical, and Internet-of-Thing systems on
the physical world, this has potentially large benefits in terms of the safety and security
of the general public.

\paragraph{Education and Outreach:}
The proposed research yields several opportunities for enhancing CS
education, recruiting new CS majors, and retaining CS students,
particularly members of underrepresented groups.  
PI Groce will work with the NAU Student ACM Chapter to present a
series of ``excursions in testing'' that introduce automated testing
to students, using feedback-driven mutation testing and Mutation-Driven-Development (MDD) on real code, including code from
media player libraries.  The work of Guzdial
\cite{Guzdial} has shown that media computation is a
potentially effective way to both recruit and retain female and
under-represented minority students in computer science.
