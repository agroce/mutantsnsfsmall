\documentclass[11pt]{article}
\usepackage[top=1in,bottom=1in,left=1in,right=1in]{geometry}

\begin{document}

\begin{center}
{\Large\sf\textbf{Collaborative Research: SHF: Small: Feedback-Driven
    Mutation Testing for Any Language: Collaboration Plan }} 
\end{center}


% The following is from the solicitation; it's the instructions for Medium/Large
% but there aren't separate instructions for Small:
%
% Since the success of collaborative research efforts are known to depend on
% thoughtful coordination mechanisms that regularly bring together the various
% participants of the project, all Medium proposals that include more than one
% investigator and all Large proposals must include a Collaboration Plan of up to
% two pages. The length of and degree of detail provided in the Collaboration Plan
% should be commensurate with the complexity of the proposed project. Where
% appropriate, the Collaboration Plan might include: 1) the specific roles of the
% project participants in all organizations involved; 2) information on how the
% project will be managed across all the investigators, organizations, and/or
% disciplines; 3) identification of the specific coordination mechanisms that will
% enable cross-investigator, cross-organization, and/or cross-discipline
% scientific integration (e.g., yearly conferences, graduate student exchange,
% project meetings at conferences, use of the grid for videoconferences, software
% repositories, etc.); and 4) specific references to the budget line items that
% support collaboration and coordination mechanisms. If a Large proposal, or a
% Medium proposal with more than one investigator, does not include a
% Collaboration Plan of up to two pages, that proposal will be returned without
% review.

The work plan for this project allows for both parallel independent progress, as
well as well-defined points of collaboration.

In Year One, PI Groce and his graduate student will both focus on the FPF-Based
Novelty Ranking and representation of mutants and test results.  This
foundational work is required before other parts of the approach can be applied.
PI Le Goues and her graduate student will focus on extending Comby with type
reasoning and LSP integration.   

For the initial work on novelty ranking and utility prediction, PI Groce's
existing mutant generation capabilities will serve as a suitable starting
point, until Comby fully replaces (or is integrated as a back-end for) {\tt universalmutator} in the toolchain.  The
PIs will clarify this integration task early, and use it to establish
the technical and social norms for collaboration between the NAU and CMU teams;
we expect to have it completed no later than mid-Year Two, to support initial
lab studies (see below).  
However, it is not a blocker for the other independent research.  
We will also use the integration of Comby into the mutation testing toolchain as
an opportunity to identify any improvements that might merit early
implementation.

In Year Two, PI Groce and his student will focus on the Mutant Utility Predictor
and Feedback Analysis, and how to integrate those with the FPF-Based Novelty
Ranking.  PI Le Goues and her student will focus on multi-language
transformation and customization, and on developing additional possible
diversity metrics for Novelty Ranking, drawing on their expertise in program
repair.

In Year Three, the demands made by the feedback-driven approach on the core
mutation engine will be much clearer, due to the relative maturity of the
user-facing components; we will improve the system accordingly.  The third year
will therefore emphasize integrating the entire system and performing thorough
evaluations, including the most comprehensive of the proposed user
studies of our approaches, as well as baselining of the utility of mutants in
general, including of Mutation-Driven-Development.
In the interest of risk mitigation and timely completion, however, we expect to
begin initial user studies earlier in the project timeline, making use of the
prototype developed no later than mid-Year Two for in-lab studies on students or
other convenience participants.  These will be primarily conducted at CMU, where
PI Le Goues and her group have support and experience in designing human
studies. 

It is likely that
by the middle of Year Two, a prototype system suitable for use real
developers and test engineers not part of the project will be available, and the
project plan aims for the graduate student to work in an industrial or
government lab setting, and try applying the process to a real system, during
the summer of Year Two.

The PIs will establish regular meetings and communication mechanisms (like a
dedicated Slack channel) for collaboration on this project, and use
GitHub code review, issues, and PRs to manage software development
collaboration in standard open-source development best-practices ways that both
PIs are highly experienced with.  Although this is the first joint
proposal from the PIs, they have been collaborating and communicating
over the past year in the context of related work on using mutation
testing to improve compiler fuzzing, with significant results,
numerous reported bugs, and
resulting contributions to open-source tools, including some additions
to Comby and the universalmutator.  The budget includes two trips to
Pittsburgh (since this will be the location of lab/human studies) for
PI Groce for in-person collaboration.
\end{document}