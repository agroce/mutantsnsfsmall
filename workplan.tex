\subsection{Work and Collaboration Plan}
\label{sec:workplan}

The work plan for this project is guided by the components in Figure
\ref{fig:flow}.  It allows for both parallel independent progress, as well as
well-defined points of collaboration.

In Year One, PI Groce and his graduate student will both focus on the FPF-Based
Novelty Ranking and representation of mutants and test results.  This
foundational work is required before other parts of the approach can be applied.
PI Le Goues and her graduate student will focus on extending Comby with type
reasoning and LSP integration.   

For the initial work on novelty ranking and utility prediction, PI Groce's the
existing mutant generation capabilities will serve as a suitable starting
point, until Comby fully replaces {\tt universalmutator} in the toolchain.  The
PIs will initial this integration task eary, and use it to establish
the technical and social norms for collaboration between the NAU and CMU teams;
we expect to have it completed no later than mid-Year Two, to support initial
labe studies (see below).  
However, it is not a blocker for the other independent research.  
We will also use the integration of Comby into the mutation testing toolchain as
an opportunity to identify and any improvements that might merit early
implementation.

In Year Two, PI Groce and his student will focus on the Mutant Utility Predictor
and Feedback Analysis, and how to integrate those with the FPF-Based Novelty
Ranking.  PI Le Goues and her student will focus on multi-language
transformation and customization, and on developing additional possible
diversity metrics for Novelty Ranking, drawing on their expertise in program
repair.

In Year Three, the demands made by the feedback-driven approach on the core
mutation engine will be much clearer, due to the relative maturity of the
user-facing components; we will improve the system accordingly.  The third year
will therefore emphasize integrating the entire system and performing thorough
evaluations, including the most comprehensive of the proposed user studies and
of (and using) Mutation-Driven-Development.
In the interest of risk mitigation and timely completion, however, we expect to
begin initial user studies earlier in the project timeline, making use of the
prototype developed no later than mid-Year Two for in-lab studies on students or
other convenience participants.  These will be primarily conducted at CMU, where
PI Le Goues and her group has support and experience in designing human
studies. 

It is likely that
by the middle of Year Two, a prototype system suitable for use real
developers and test engineers not part of the project will be available, and the
project plan aims for the graduate student to work in an industrial or
government lab setting, and try applying the process to a real system, during
the summer of Year Two.

The PIs will establish regular meetings and communication mechanisms (like a
dedicated Slack channel) for collaboration on this project.  Although this is a
new collaboration for the PIs, they are both experienced in successful
cross-institutional collaborations.
