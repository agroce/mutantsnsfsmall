\subsubsection{Work Plan}

For this project, a simple plan should suffice, defined by components
of Figure \ref{fig:flow}.  In Year One, PI Groce and the graduate
student will both focus on the FPF-Based Novelty Ranking and
representation of mutants and test results.  In Year
Two, both will focus on the Mutant Utility Predictor and Feedback
Analysis, and how to integrate those with the FPF-Based Novelty
Ranking.  In Year Three, the focus will shift to the Mutant Generator
itself, and the problem of any-language mutant generation.  The third
year will also emphasize integrating the entire system and performing
more thorough evaluations.

\subsubsection{Evaluation Plan}

One automated way to evaluate a novelty ranking is to use automated
testing to generate multiple tests to kill each mutant in ranked order.  A good ranking will mean
that each additional mutant is unlikely to be killed by the killing
tests for any previous mutants.  A similar, but more robust, measure of mutant similarity
is how much adding using test to kill one mutant as the seed in a
fuzzer~\cite{aflfuzz,libfuzzer} improves time required to kill the other
mutant, on average.  Unfortunately, these measures cannot effectively
measure ``real distance'' if one of the mutants is equivalent.