\subsection{Work Plan}
\label{sec:workplan}

For this project, a simple plan should suffice, defined by components
of Figure \ref{fig:flow}.  In Year One, PI Groce and the graduate
student will both focus on the FPF-Based Novelty Ranking and
representation of mutants and test results.  This foundational work is
required before other parts of the approach can be applied.  For work
in the first two years, the existing mutant generation capabilities
will likely suffice, although it is possible that some improvements
will be discovered to provide enough benefit to merit early
implementation.  In Year
Two, both will focus on the Mutant Utility Predictor and Feedback
Analysis, and how to integrate those with the FPF-Based Novelty
Ranking.  In Year Three, the focus will shift to the Mutant Generator
itself, and the problem of any-language mutant generation.  By this
time, the demands made by the feedback-driven approach on the core
mutation engine will be much clearer, due to the relative maturity of
the user-facing components.  The third
year will also emphasize integrating the entire system and performing
more thorough evaluations, including of (and using) Mutation-Driven-Development.  It is likely that by the middle of Year
Two, a prototype system suitable for use by real developers and test
engineers not part of the project will be available, and the project
plan aims for the graduate student to work in an industrial or
government lab setting, and try applying the process to a real system,
during the summer of Year Two.
