
\subsection{Evaluation Plan}
\label{sec:evalplan}

We propose a multi-prong evaluation strategy, including (1) proxy metrics
suitable for evaluating program transformation and mutation testing, and the
individual components of the overall research program, (2) a series of lab
studies on tool usability and mutation-driven development as a paradigm
generally, and (3) qualitative experiences using and evaluating the tool,
ideally in collaboration with industrial partners.

\paragraph{Proxy metrics for individual research components.}
Just computing a mutant kill matrix for a good test suite can also
partially evaluate novelty rankings, by ranking
killed, rather than unkilled mutants; the experiment even
realistically represents an early stage of test suite construction by
feedback-driven mutation testing, especially if the mutants are only
killed by a relatively small set of tests.  If
mutant X and mutant Y are both highly ranked, but killed by a very
similar set of tests, this indicates a possible problem with the
measure.  In real-world feedback-driven mutation testing, it is highly
desirable not to compute the full kill matrix for all mutants, but
for evaluation purposes, determining the extent to which kill
bitvector similarity agrees with FPF distance metric similarity serves
as a basic, if quite imperfect, sanity check on the novelty ranking.
Another automated way to evaluate a novelty ranking is to use automated
testing to generate multiple tests to kill each mutant in ranked order.  A good ranking will mean
that each additional mutant is unlikely to be killed by the killing
tests for any previous mutants.  A similar, but more robust, measure of mutant similarity
is how much adding a test that kills one mutant as the seed in a
fuzzer~\cite{aflfuzz,libfuzzer} improves time required to kill the other
mutant, on average.  Unfortunately, these measures cannot effectively
measure ``real distance'' if one of the mutants is equivalent.

Evaluation of the mutant generator can also be partly automated, by
comparing the output set of mutants to that of other tools, to ensure
no important mutants are omitted; the regular-expression based
approach will probably generate valid mutants not generated by other
tools, since it aims at a rich operator set, in accord with the
suggestions of multiple previous papers on detecting faults via
mutation~\cite{just2014mutants,gopinath2017mutation}.  Efficiency
gains (e.g., removal of invalid or equivalent-by-construction mutants) can
be measured by simple, traditional measures.
\textbf{FIXME: RVT, please add metrics/experiments for the type stuff?}

\paragraph{Lab studies.}  
We will conduct pilot and lab studies to evaluate tool design
and usability. 
These studies will generally involve asking programmers to perform a set of
constructed software testing, bug finding/fixing, or maintenance tasks, with or
without a tool.  The results of such studies can be evaluated using both
quantitative measures (like time and success rate on the provided tasks) as well
as qualitative coding~\cite{something} and theory-building techniques to surface
important challenges or benefits to the tooling or underlying Mutation-Driven
Development methodology.  We can enhance external validity by basing the
programming tasks on common forum postings, as we have done previously for
studies of debugging challenges in particular
contexts~\cite{frameworkDebugging}; other researchers have demonstrated this
type of methodology useful for tool pilot studies~\cite{sunshineDocumentation}.
We may also construct implementation tasks around small, but easy-to-get-wrong
code problems like binary search and AVL trees, especially when evaluating the
potential benefits or challenges to an MDD approach to programming.  

We may make use of ``dummy'' or ``Wizard of Oz'' versions of elements of the
framework to isolate the effects of specific features, like the FPF-based
novelty metric, or various choices for feedback elicitation or analysis. 
We will use
% verbal reports as data
think-aloud protocols~\cite{thinkaloud}, in which participants are instructed to
continuously verbalize what they are thinking/attempting. This method helps the
examiners determine why participants behave in particular ways, and to identify
and isolate sources of confusion.  These barriers can then be used to improve
the undelrying research technique.  To mitigate the risk of highly varying
programmer skill, we will use a counterbalanced, within-subjects design,
exposing each participant to both experimental and control conditions.  The
design just outlined has been profitably used in many
experiments about programming tools and
methodology~\cite{Endrikat2014,Stylos07}.  PI Le Goues has experience in both
lab studies involving think aloud protocols~\cite{frameworkDebugging} as well as
qualitative analysis generally~\cite{testingUnderReview}, and will lead the
design and execution of these lab studies at CMU.
% Stefan Endrikat, Stefan Hanenberg, Romain Robbes, and Andreas Stefik. How do
% api documentation and static typing affect api usability? In International
% Conference on Software Engineering (ICSE), 2014. 
% Jeffrey Stylos and Steven Clarke. Usability implications of requiring
% parameters in object’s constructors. In International Conference on Software
% Engineering (ICSE), 2007. 

\paragraph{Experiential, qualitative evaluations.}
Evaluation for this proposal includes both human-performed assessment
via trying to use feedback-driven mutation for actual test improvement
tasks and automated evaluations with more objective criteria, but a
weaker relationship to the actual goal of helping users quickly find
the most important unkilled mutants.  For the human portion, informal
evaluation will be performed by the research team itself, using known
programs with known testing weaknesses; this will help tune the
approach and experiment with new ideas.  However, for more advanced
assessment, expert users outside the team will be offered the chance
to use the system once it is in suitable shape.  In the past, PI Groce
has worked with IBM Distinguished Engineer Paul E. McKenney, a
prominent Linux kernel developer, on using mutants to improve kernel
test suites, and has discussed similar efforts with Richard Hipp, the
lead developer of the SQLite database, a rather famously well-tested
program, with some resulting improvements to both test suites.  Other
potential users with whom PI Groce has a working relationship include  NASA/JPL engineers working on upcoming CubeSat
missions, colleagues working on the DeepState~\cite{DeepState}
parameterized unit testing interface to fuzzers and symbolic execution
engines, and the developers of {\tt pyfakefs}.  This type of
evaluation is, of necessity, somewhat qualitative.  A more unbiased retrospective
version that retains the core element of human rating of the value of mutants can be performed by examining actual mutants that resulted in
improvements to the rcutorture~\cite{rcutorture} tests for the Linux
kernel and the {\tt pyfakefs} tests in previous
work~\cite{groce2018verified}, and comparing the useful mutants to the
highly ranked mutants:  could the highly ranked mutants have motivated
the key improvements to the test suites?


\paragraph{Mutation-Driven Development.}
The evaluation of techniques and tools developed in this proposal will
also serve a dual purpose with respect to
Mutation-Driven-Development.  Using an MDD approach to implement
various small, but easy-to-get-wrong, code projects, such as binary
search and AVL trees, will make it possible both to see how effective
the tools for feedback-driven mutation testing are, and to see how effective an MDD approach to development is.  In
addition, using various versions of actual TDD efforts, with the
associated test suites for each step of development, it should be
possible to evaluate how much additional testing power MDD would have
required at each step of development.
