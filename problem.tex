Mutation testing is: <FIXME>.  Significant prior work, dating back to the
FIXMEs, aims to render mutation testing useful for constructing high quality
test suites and, by extension, software. 
Most of this previous mutation testing
research focuses on computing a mutation score, a measure of adequacy for a
given test suite.  However, this is computationally intensive for realistic
projects, because it requires requires running many tests on many modified
versions of a software system.  Reducing that computational cost is thus a major
thrust of mutation testing research~\cite{jia2011analysis}.   

Although test suite adequacy is certainly a useful thing to measure, the most
important goal of mutation testing\,---\,and indeed its original use
case~\cite{something}\,---\,is to help \emph{improve} a test suite.  For this
purpose, either a score or a list of all unkilled mutants generated across an
expensive mutation campaign is not useful for practicing engineers.  An
undifferentiated list of unkilled mutants contains many uninteresting or
redundant mutants, and a much smaller number of actionable mutants that are
maximally useful in guiding test improvement.  Thus, examining all unkilled
mutants is only practical for formal verification efforts or critical software
systems with high-powered test suites.  But even in these settings, examining
surviving mutants produced by modern mutation testing campaigns is
time-comsuming and unpleasant, as PI Groce and co-authors noted in prior work on
using mutants to drive formal verification and automated
testing~\cite{groce2015verified,groce2018verified,mutKernel}.


Our goal is to enable \emph{Just Enough Mutation Testing}: We propose a mutation
testing framework that identifies and interactively presents a few, very
different, ranked mutants, and then works with the user to use those mutants to
effectively improve the program, the test suite, or both.  Our proposal requires
several fundamental novelties:
\begin{itemize}
\item \textbf{Efficient, any-language mutation testing.}  The most widely used
  mutation testing tool in the real world is PIT~\cite{pittest}, which targets
  Java bytecode.  There are recent attempts to provide the same kind of support
  for other languages, especially C, by targeting LLVM IR~\cite{HaririLLVM}.
  This poses several problems for real-world mutation testing.  First, bytecode-
  or IR-level mutation works well to compute a score for a test suite, but is
  not not suitable for presentation to developers or test engineers, who need to
  reason about a mutant's implications for their source or test code (i.e., Java
  developers think in terms of Java, not compiled bytecode).  Even when
  possible, translation may not help: a bytecode-level mutation may not have a
  simple source-level equivalent, especially if the bytecode has been optimized.
  Second, features that help identify semantically similar mutants are hard to
  identify at the bytecode level.  Even if the mutant is, for example, a
  constant replacement in one case and an arithmetic operator replacement in
  another, the fact that both take place inside an argument to a logging
  function with an {\tt INFO} argument may be enough to predict that their
  effects are redundant.  Finally, bytecode-level mutation is highly
  language-specific, leaving out popular languages like Python, Ruby, or Go, not
  to mention project-specific Domain Specific Languages (DSLs)~\cite{Fow10}.
  This is a problem for tool uptake and applicability: the vast majority of
  real-world software projects are written in multiple languages~\cite{Ray2014}.
  We propose novel mechanisms for \emph{efficient, any language mutation
    testing} based on our novel recent work on language-agnostic declarative
  program transformation using parser combinators~\cite{rvt-ppc}.  This
  technique operate at the source level and produces mutants and mutations that
  are easy for developers to understand. 
\item \textbf{Mutant prioritization and selection by predicted payoff.}
An unkilled mutant is, conceptually, very similar to a failing test.
With a larger number of unkilled
mutants, the problem becomes one very much like the bug triage or ``fuzzer
taming'' problem in random testing/fuzzing~\cite{PLDI13,SemCrash,vantonder-ase18}:  a user wants
to quickly find mutants that indicate the most important ``holes'' in a testing
or verification effort, and act on those most-critical gaps, possibly revealing
faults in the System Under Test (SUT). 
Fuzzers tend to produce very large numbers of failing tests for a much
smaller number of distinct bugs.  Finding the set of distinct bugs,
and identifying important bugs that need to be fixed immediately is
difficult, because the important bugs may be represented by only one
or two failing tests in a set of thousands of failing tests, most of
which are duplicates. 
Users do not (usually) care much
about finding the group of all tests failing due to a fault, or the
set of all mutants killable by the same extension to a test suite or
generator, but about seeing \emph{many very different test failures}
or \emph{many
  different unkilled mutants} quickly, to maximize the chance of
discovering the most important faults or holes in a testing effort.
However, current mutation testing approaches make
no real effort, with few exceptions~\cite{MutGoogle,FaRM} to prioritize mutants,
and none are based on a user-centered feedback loop, where the user and mutation
testing framework interact to improve a test suite, automated test generator, or
verification harness --- and the SUT.  Other than (arguably) some efforts to
incorporate dominance results~\cite{MutQuality}, no mutation testing approaches
currently suggest any more sophisticated way to maximize the novelty of
presented mutants than stratified sampling~\cite{gopinath2017mutation};
stratified sampling does not aim at semantic novelty, and can present many
mutants from the same class, if applied at the method level, even if those
mutants are highly similar in impact.  Other work~\cite{gopinath2015howhard}
proposes random sampling as the most effective way to select mutants.
Unfortunately, when an important class of unkilled mutants has only a few
members, random sampling is almost guaranteed to fail to present any of them.  
We propose to adapt clustering optimization techniques based on the idea of
novelty~\cite{Gonzalez85} to the problem of mutant selection, informed by a set
of novel diversity metrics selected for this domain. 
\item \textbf{User feedback elicitation and analysis.}
A
user's feedback about the most critical-to-test aspects of the code, or hard
work examining some mutants, has no influence on the kinds of sampling currently
proposed in the mutation testing literature.  
Even creating simple clusters of
mutants that are not killed due to the same underlying omission in tests
requires manual effort, with users, e.g., writing a Python script scanning
mutants for certain strings and assuming all mutated code with that string is
part of the same ``equivalence class.''  This is a tedious and error-prone
process, and only even possible once a ``kind'' of unkilled mutant is
discovered, largely by ad hoc scanning of the list of unkilled, uncategorized,
mutants.  
A constantly-updated ranking of mutants likely to be useful to examine
also helps provide a stopping rule other than patience, time available, or
``every last mutant'':  since mutants are ranked by likely payoff, once a user
has examined several mutants in a row without benefit, or mutants are highly
similar in behavior to other mutants, a user may reasonably stop, knowing that
the low-hanging fruit have probably all been picked.  Finally, a feedback-driven
mutation testing approach provides a new way to improve the efficiency of
mutation testing:  even for a very large project, it only has to build and
execute a small set of mutants, because it only runs the test suite on mutants
currently predicted to be of likely interest to the user.  
We propose ``feedback-driven'' mutation analysis that elicits and incorporates
user input on mutants, tests, and the SUT, supporting a concise, updating list
of mutants to inspect based on expected utility or payoff for the user.
\end{itemize}


\subsection{Problem Statement}

Overall, this project aims to make the use of program mutants practical in
non-research settings, in a way that meets developers' actual needs: to make it
possible for someone creating or enhancing a test suite to (1) use ``just
enough'' mutation testing for their needs, maximizing benefit gained in exchange
for work performed, and to (2) work in any programming language without worrying
about the quality of tool support, and while providing intuitive source-based
mutants and easy customization.  This project also aims to make use of the insights of
Test-Driven-Development (TDD), and proposes using mutation testing to move
beyond a paradigm where developers build a series of tests narrowly tailored to
steps in development, and use Mutation-Driven-Development (MDD) to build
automated test generators or verification harnesses that handle not only
anticipated problems imagined during development, but problems not anticipated
by developers.  In addition to traditional manual testing, this proposal targets
property-driven testing and full formal verification of software components, in
order to be practical in the future, when software systems will often be so
safety- or mission- critical that even ``good'' manual testing is simply not an
acceptable approach. 


\begin{framed} {\bf Problem:} Develop highly automated methods and tools that
  allow the practical application of mutation testing to real-world software in
  a feedback-driven way, where user and mutation testing framework cooperate to
  improve testing efforts, while minimizing user effort and maximizing the
  ability to quickly find the most important weaknesses of testing or
  verification.
\end{framed}

% \begin{framed}
% {\bf Problem:}  Provide high-quality user-friendly mutation generation methods to be used in a flexible but efficient mutation testing framework that can be applied to any source language, with support for easily adding custom operators or mutating DSLs.
% \end{framed}


\subsection{PI Qualifications}

PI Groce has been a user of, and contributor to, mutation testing tools for many
years.  He combines a long research track record in software testing, including
mutation testing, with actual experience testing critical software systems at
NASA's Jet Propulsion Laboratory.  PI Groce's long-running interest in improving
mutation testing arises from frustration in his efforts to apply mutation to the
Mars Science Laboratory's flight software, in particular to the file
system~\cite{ICSEDiff,CFV08,AMAI}.  This practical orientation informs his
recent work on using mutation testing in a falsification-driven approach to
improving verification and automated testing
efforts~\cite{groce2015verified,groce2018verified,mutKernel}.  PI Groce has
extensive experience in developing mutation tools for new
languages~\cite{le2014mucheck,muupi,regexpMut}, including the first reliable
tools for mutation of Haskell, Python, and Swift, as well as in user-facing
(vs. researcher-oriented) automated software testing
tools~\cite{tstlsttt,DeepState}.  He additionally has expertise in driving
testing of machine learning systems through user
interaction~\cite{EndUserMistake,OnlyOracle}. 

PI Le Goues is an expert in applied program analysis, program transformation,
and testing, most relevantly through her pioneering work in automated program
repair (including heuristic~\cite{genprog1,another-genprog} and
semantic~\cite{s3,Ke15ase} dynamic approaches, and approaches guided by
static analysis~\cite{footpatch}).  She has significant experience
with testing, mutation testing for fault localization and program repair
particularly~\cite{ssbse,faultloc}, and the challenges of syntactic program
modification and transformation.  In particular, her recent work has developed
novel mechanisms for efficient and expressive language-agnostic syntactic
program transformation~\cite{rvt-ppc}, with applications for, e.g., program
repair~\cite{footpatch}, fuzz test triage~\cite{vantonder-ase18}, and static analysis
customization~\cite{icse-underreview}.  
