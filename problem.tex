The core problem this project aims to address is making the use of program mutants practical in non-research settings, in a way that meets developers' or test engineers' needs; that is, making it possible for someone creating or enhancing a test suite, or developing code and test suite in tandem, to (1) use ``just enough'' mutation testing for their needs, maximizing benefit gained in exchange for work performed, (2) easily use custom mutation operators that target their specific software development task, and (3) work in any programming language without worrying about the quality of tool support provided for mutation testing.  More generally, this project aims to make use of the insights of Test-Driven-Development (TDD), and proposes using mutation testing to move beyond a paradigm where developers build a series of tests narrowly tailored to steps in development, and use Mutation-Driven-Development (MDD) to build automated test generators or verification harnesses that handle not only anticpated problems imagined during development, but problems not anticipated by human insight, discovered using mutation-based analysis.  In addition to traditional manual testing, our approach targets both highly-general property-driven testing and even full formal verification of software components, in order to be practical in the future, where software systems will often be so safety- or mission- critical that even ``good'' manual testing is not an acceptable approach to ensuring correctness, security, and reliability.



\subsubsection{PI Qualifications}

PI Groce ...