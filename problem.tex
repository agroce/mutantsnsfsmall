The core problem this project aims to address is making the use of program mutants practical in non-research settings, in a way that meets developers' needs; that is, developers can use ``just enough'' mutation analysis for their needs, maximizing benefit gained in exchange for work performed, easily use custom mutation operators that target their specific software development task, and work in any programming language without worrying about the quality of tool support provided for mutation testing.  More generally, this project aims to enable taking the insights of Test-Driven-Development (TDD), and propose using mutation testing to move beyond a paradigm where developers build a series of tests narrowly tailored to steps in development, and use Mutation-Driven-Development (MDD) to build automated test generators that handle not only anticpated problems imagined during development, but problems not anticipated by human insight, but derived from mutation-based analysis.

\subsubsection{PI Qualifications}

PI Groce ...