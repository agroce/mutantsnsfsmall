\paragraph{Our Vision: A Day in the Life of a Systems Programmer}

\begin{framed} \emph{Our final research goal, to enable cooperation between mutation, developer intuition, and fuzzing and other automated testing systems, in a seamless fashion, is most easily expressed in a scenario, rather than as an explicit research program or set of questions.  It depends on the other proposed work, and on integration of the final products with automated testing systems.  Both PIs are expert in the use and enhancement of such tools.  We have italicized novel aspects of Laura's work that relate to this project.}
  \end{framed}

Laura is developing a data-management system for a NASA/JPL Mars rover.  She is responsible both for a catalog process that handles data products and critical configuration files and a low-level file system for custom, radiation-hardened, flash software.  Laura has, in the process of developing the code, also developed a set of manually constructed unit tests \emph{and harnesses supporting automated test generation for the system, using IDE-supported MDD}.  She employs static analysis tools, with additional custom rules developed by the rover software team and the NASA facility's software tooling experts.  Laura has also \emph{added a few custom rules that eliminate certain file system API mutants not easily detected dynamically}.

Today, Laura is continuing an ongoing effort at changing functionality to the catalog and the file system, rewriting the ``rename'' call to restrict behavior more than the POSIX standard.  After reviewing the risks of arbitrary rename, the software team has agreed this protects the integrity of the file system while retaining all functionality ground operators might need to fix problems.  As part of this revision, Laura realized that with these restrictions a change to the core rename atomicity scheme could make mounting and checking the file system both faster and less complex (thus likely less buggy).
Laura reaches a point where she believes her change is fully implemented and the tests have been revised to check the changed behavior.  \emph{She looks at a panel in her IDE showing Test Effectiveness, and notices that there are four interesting Missed Bugs for her to examine.  She clicks on the top one, and the IDE focuses on a line of code in the rename function.  The display shows a change to that line, and says that altering the line in the way shown changes the file system binary but is not detected by any existing manual test or static analysis rule, and has not been detected in two minutes of fuzzing, across multiple cores, or by any stored fuzzing-generated test.  Because Laura has been tuning the MDD-alert system via feedback, all four mutants are all highly interesting to her, and very different from each other; this one is the most interesting of all.}

Laura thinks for a while about the change made, which involves passing a flag to the call that disables an expensive ``sanity check'' on header values for a file, to be used in contexts where the header has just been written and checked.  She thinks the current context is unsafe, and the check is required.  The danger arises from certain hardware failure modes, so she goes to the hardware emulator code that she uses to run tests without access to the rover testbeds, and \emph{requests the system to generate tests that target not just the mutated line of code (the system has already generated a number of these for her to examine), but the additional line with the relevant simulated hardware failure injection.  The fuzzers run in a targeted mode for a few minutes, and show Laura traces.}  She runs some of these through a debugger, goes to the blackboard and calls in a colleague to discuss her reasoning.  They agree the check is indeed not needed, which will result in a small optimization in the file system.  \emph{Given the process she uses for developing her code, and the high quality of automated tests associated, Laura feels confident, given the argument, in optimizing the code.  The system now informs her that a very different mutant has the highest priority among uncaught bugs.  Of course, the system couldn't detect the obvious mutant to the new version of the code, enabling the check, but Laura long ago informed the system that adding the check was never going to break any tests (the check is too cheap to break performance tests in any one instance.  It therefore won't trouble her with this problem, now.  A small symbol attached to the parameter on the line of code does allow Laura to see that this line has an ignored unkilled mutant, which she can inspect if she is worried about the code.}

    \emph{Because Laura has been using the ``MDD'' process all along, her additional confidence tests will detect any problems with the optimization is not just based on mutation analysis of her current code}.  Instead, it is \emph{based on mutants of all the versions of the design and functionaity she has implemented}; if a problem is representable by a mutant of code no longer present (the older rename behavior, for example), then \emph{while not forming a basis for mutant alerts now, that behavior, in terms of API calls and hardware simulation choices, will be permanently in place in the stored corpus of fuzzing tests or manually constructed tests to address mutants.  So long as Laura allows the tool to guide her to always kill all interesting mutants it advises her about, she will have these ``mutant-regression'' tests as a ``memory'' of the varying possibilities of her design, and the assurance tasks associated with design perimeters.}
