\section{Results From Prior NSF Support}

PI Groce has received support as PI or co-PI from three NSF grants.
The most relevant and recent is ``Diversity and Feedback in Random
Testing for Systems Software'' (CCF-1217824,
\$491,280, 9/2012--9/2017), a collaborative proposal with John Regehr at the
University of Utah.
{\bf Intellectual Merit:} 
%CCF-1217824 
%includes a preliminary exploration of how to
%``tame'' fuzzer output~\cite{PLDI13fuzzing}, a problem addressible with better mutation
%infrastructure.  
A key result from CCF-1217824
is the development of a strategy for creating ``quick tests''~\cite{icst2014}, which won the
Best Paper award at ICST 2014 for showing tests thus
reduced can serve as effective regression tests or seeds for
symbolic execution~\cite{stvrcausereduce, issta14}.  Moreover, benefits do not depend on 100\%
preservation of a property~\cite{AlipourETAL16TestReduction}.   Other results include an overview of the value of coverage in
testing experiments~\cite{Onward14} and exploration of how individual
test features impact the coverage and fault detection statistics of
random tests~\cite{helphelp}.  
%The basic swarm testing approach has
%been extended to allow production of focused random tests targeting
%particular code~\cite{DirectedSwarm}.  
All of these results used
mutation testing.  {\bf Broader Impacts:} CCF-1217824 has contributed to the discovery of previously
unknown faults in multiple open-source and commercial software
systems, including core compilers and system libraries.  The
development of the central swarm testing techniques
has furthered many efforts to improve
the quality of compilers, including LLVM and GCC, and to test core language
tools in
general~\cite{ZhendongPLDI14,beginnerluck,dewey2015fuzzing,le2015randomized}. {\bf
  Research
Products:} Several publications resulted from this grant, including
those cited above and numerous others ~\cite{Onward14,PLDI13,issta14,icst2014,helphelp,DirectedSwarm,stvrcausereduce,tstlsttt,ISSTA15,AlipourETAL16TestReduction,tstlsttt,NFM15},
along with three PhD theses.  Source code~\cite{swarmtools,TSTL}  is
available on GitHub.