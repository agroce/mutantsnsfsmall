\section{Results From Prior NSF Support}

PI Groce has received support as PI or co-PI from three NSF grants.
The most relevant and recent is ``Diversity and Feedback in Random
Testing for Systems Software'' (CCF-1217824,
\$491,280, 9/2012--9/2017), a collaborative proposal with John Regehr at the
University of Utah.
{\bf Intellectual Merit:} 
%\subsection{Intellectual Merit}
%CCF-1217824 
%includes a preliminary exploration of how to
%``tame'' fuzzer output~\cite{PLDI13fuzzing}, a problem addressable with better mutation
%infrastructure.  
The results of CCF-1217824 included a preliminary exploration of how to
``tame'' fuzzer output, a problem also central to this proposal~\cite{PLDI13}.  In previous work, the goal was to find an algorithm
for using hand-chosen distance metrics to identify bugs in tests.
%A key result from CCF-1217824
%is the development of a strategy for creating ``quick tests''~\cite{icst2014}, which won the
%Best Paper award at ICST 2014 for showing tests thus
%reduced can serve as effective regression tests or seeds for
%symbolic execution~\cite{stvrcausereduce, issta14}.  Moreover, benefits do not depend on 100\%
%preservation of a property~\cite{AlipourETAL16TestReduction}.   Other results include an overview of the value of coverage in
%testing experiments~\cite{Onward14} and exploration of how individual
%test features impact the coverage and fault detection statistics of
%random tests~\cite{helphelp}.  
%The basic swarm testing approach has
%been extended to allow production of focused random tests targeting
%particular code~\cite{DirectedSwarm}.  
Many other key results~\cite{DirectedSwarm,issta14,helphelp,Onward14} used mutation testing as an
evaluation method.
{\bf Broader Impacts:} 
%\subsection{Broader Impacts}
CCF-1217824 has contributed to the discovery of previously
unknown faults in multiple open-source and commercial software
systems, including core compilers and system libraries.  The
development of the central swarm testing techniques
has furthered many efforts to improve
the quality of compilers, including LLVM and GCC, and to test core language
tools in
general~\cite{ZhendongPLDI14,beginnerluck,dewey2015fuzzing,le2015randomized}. 
{\bf Research Products:}
% \subsection{Research Products}
Publications resulting from CCF-1217824 were numerous~\cite{Onward14,PLDI13,issta14,icst2014,helphelp,DirectedSwarm,stvrcausereduce,tstlsttt,ISSTA15,AlipourETAL16TestReduction,tstlsttt,NFM15},
along with three PhD theses.  Source code for software systems
developed or enhanced during CCF-1217824~\cite{swarmtools,TSTL}  is
available on GitHub.

Le Goue's most closely-related prior NSF grant is \textbf{FIXME: number}, CAREER
Quality Matters: Dynamic, Static and Proactive Analyses for Automated Program
Repair.  \textbf{Intellectual Merit:} The results of this award have so far
included novel techniques for static program repair~\cite{footpatch}, an initial
exploration of diversity-enhancing dynamic repair
techniques~\cite{undergrads-gi}, and the Comby tool and associated mechanism for
declarative, language-agnostic program transformation using parser parser
combinators~\cite{rvt-ppc}.  Neither mutation testing nor language-agnostic
program transformation primitives are the core focus of the prior award,
however, which instead focuses specifically on developing push-button automatic
program repair techniques. This new proposal seeks to extend the work on
declarative program transformation, with a particular application to mutation
testing. \textbf{Broader Impacts:} The award has so far supported two REU
students, including one member of an underrepresented group in computing and
another student without access to traditional research opportunities at their
home institution.  Sophia Kolak recently gave a well-received talk at ROSCon
2019 on her summer research; Zhen Yu Ding is the first author on a published
paper on his work~\cite{gi-undergrads}.  Both are continuing their research with
the PI and have expressed plans to attend graduate school in Computer Science.
Additionally, the tools and techniques developed in the research so far are open
source and available on GitHub, and the PPC work has been presented to a mixed
audience of academics and developers at StrangeLoop~\cite{strangeloop}, an
important form of outreach to the engineering community.
