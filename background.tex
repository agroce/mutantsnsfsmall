\section{Background and Preliminary Research}

\subsection{Furthest Point First and Fuzzer Taming}

An unkilled mutant is, conceptually, very similar to a failing test.
It presents information of possible relevance to a developer or test
engineer.  The mutant or test \emph{may} indicate the presence of a
previously unknown fault that needs to be fixed, either in the SUT or in the test suite/test
generator.  It may indicate the presence of a previously unknown fault
of less importance.  It may also indicate an even less interesting
result:  an equivalent mutant or 
a failure of an inherently flaky test.  Or, in many cases, an unkilled
mutant or failing test may contain information that is either
important or unimportant, but is uninteresting because \emph{it
  duplicates information already presented for understanding.}  While
examining an equivalent mutant is not always useless (e.g., it may indicate
an opportunity for refactoring or improving the efficiency of code~\cite{ivankovic2018industrial,groce2018verified}), examining a mutant that is
equivalent to or extremely similar to an already-understood mutant is almost never
worthwhile --- even if the original mutant provided important,
actionable information.  That information has already been
incorporated into the development or testing process.

Fuzzer taming~\cite{PLDI13} was a solution we proposed to the problem
of triaging test failures in automated test generation.  In compiler
testing and other fuzzing applications, a core usability issue is that
tools tend to produce very large numbers of failing tests for a much
smaller number of distinct bugs.  Finding the set of distinct bugs,
and identifying important bugs that need to be fixed immediately is
difficult, because the important bugs may be represented by only one
or two failing tests in a set of thousands of failing tests, most of
which are duplicates.