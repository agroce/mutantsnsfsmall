\documentclass{article}

\begin{document}


\begin{center}
{\Large\sf\textbf{SHF: Small: Feedback-Driven Mutation Testing for Any Language}}
\end{center}


The primary equipment used in this project will be standard workstations and commodity computing equipment; e.g., for development and analysis work, high-end workstations or even laptops will suffice, in line with current development practices in industry and research; one purpose of this project is to make mutation-driven testing feasible in a context such as a developer working on a laptop, with no other computational resources. Northern Arizona University has the usual high-end network and computing infrastructure for standard computing uses.  Should the need for more ambitious computing requirements arise, Northern Arizona University provides access to powerful additional computing infrastructure, also available to this project.   If any more unusual computing needs arise, NAU has several ways to meet these needs, discussed below.

AEGIS is a private cloud computing facility being developed collaboratively by the University of Arizona (UA), Northern Arizona University, and Arizona State University (ASU) with the support of the Arizona Board of Regents. AEGIS is based on the NSF-supported iPlant (now CyVerse) cyberinfrastructure (http://www.iplantcollaborative.org) ) and provides cloud-based services to researchers and educators in the environmental and ecological sciences. It uses the NSF-supported iRODS (www.irods.org) distributed storage service and the OpenStack open-source cloud operating system, enabling researchers to develop custom images that support scientific workflows and deploy them as virtual machines in to the AEGIS cloud for execution. NAU’s AEGIS hardware platform is networked with servers at UA and ASU and consists of a storage server with 2 X Intel Xeon E5-2620 2.00 GHz 8-core processors, 64 GB of DDR3 ECC memory, and 96 TB of disk storage in a RAID-6 configuration.

NAU’s Monsoon high-performance computing facility is a capacity-type, Linux-based computer cluster with 2860 Intel Xeon cores, 24TB of memory, and 16 NVIDIA GPUs: K80, and P100. It has been designed to be flexible and handle a diverse set of research requirements. 103 individual systems are interconnected via FDR Infiniband at a rate of 56Gbps and $<.07$us latency. Cluster nodes have access to 1.3PB of shared storage of type scratch (lustre), and long-term project space (ZFS). Monsoon has a measured peak performance of 107 teraflops. It is housed in the NAU ITS Server Co-Location Room, which contains the necessary power and other facilities requirements such as a non-water fire suppression system.

\end{document}
